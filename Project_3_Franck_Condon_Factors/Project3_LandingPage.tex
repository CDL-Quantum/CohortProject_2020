\documentclass[12pt]{article}

% packages
\usepackage{setspace}
\usepackage{array}
\usepackage[margin=0.75in]{geometry}
\usepackage{amsmath,bm}
\usepackage{amssymb}
\usepackage{bbold}
\usepackage{physics}
\usepackage{xcolor}
\usepackage{indentfirst}
\usepackage{enumerate}
\usepackage{mathtools}
\usepackage{fancyhdr}

\pagestyle{fancy}
\fancyhf{}
\rhead{Creative Destruction Lab}
\lhead{Introductions to Projects}
\rfoot{Page \thepage}

\allowdisplaybreaks

\title{Project 3: Calculating Franck-Condon Factors}

\begin{document}

\maketitle

\thispagestyle{empty}

\subsection*{Motivation}

Spectroscopy is the study of how light and matter (atoms and molecules) interact.
Light can be absorbed by matter (\textit{absorption}) or matter can emit light (\textit{emission}).
It turns out that spectroscopy is scientists' main tool for discovering properties of molecules (e.g. their molecular structure, bond strengths, etc.) and for chemical identification. Picture for instance an astronomer taking spectroscopic measurements from a gas cloud located millions of light years away from the Earth. To determine what chemicals this gas cloud comprises of, they certainly cannot travel there and take a physical sample of the gas cloud. The astronomer is tasked with utilizing spectroscopic theory (i.e. models wherein numerical studies can be carried out efficiently) to explain what they see.

Countless other applications like drug discovery, magnetic-resonance imaging (MRI) and climate science rely heavily on spectroscopic methods and theory. This week, you'll familiarize yourself with calculating Franck Condon Factors (FCFs), which are useful in studying {\it vibronic} transitions in molecules. You'll also get to compare your calculations to real experiments.

\subsection*{Your Tasks}

\subsubsection*{Task \#1}

In this task, you will calculate Franck-Condon Factors for H$_2$-H$_2^+$ using the harmonic oscillator approximation and compare to a real experiment!
You are provided with a \texttt{python} notebook called \texttt{Task1.ipynb} which calculates all of the information you require. This code takes as input:
\begin{enumerate}
    \item Upper bound of ground and excited vibrational states allowed for transitions
    \item Reduced mass of H$_2$ molecule (same as H$_2^+$)
    \item Frequency of the H$_2$ molecule  (ground state, 0)
    \item Frequency of the H$_2^+$ molecule (excited state, p)
    \item Equilibrium bond lengths (the difference is the displacement)
    \item Difference between potential minima of the electronic states (ionization energy)
\end{enumerate}
This code outputs:
\begin{enumerate}
    \item Franck-Condon Factors for each pair of ground and excited vibrational states up to a threshold, using the n$_{H_2}$ = 0, n$_{H_2^+}$ = 0 Franck-Condon Factor as the reference.
\end{enumerate}
Your task will be to plot the spectra of at least 10 transitions with a relative Franck-Condon Factor greater than 1\%, where the FCF is the intensity and the spectral position (SP) is defined as:
\begin{equation}
    \text{SP} = IE + E_{H_2^+} - E_{H_2}
\end{equation}

\noindent All of these energies are calculated for you ($E_0$ and $E_p$). You will need to store the Franck-Condon Factors and plot them. You may also need to adjust the bounds of allowed ground and excited vibrational states to reach 10 transitions. 

Congratulations, you have now successfully predicted the Franck-Condon Factors of H$_2$ - H$_2^+$! Figure \ref{fig:h2_spectrum} shows the photoionization spectrum for H$_2$. How does it compare to your calculation using the harmonic oscillator approximation?

\begin{figure}
    \begin{center}
        \includegraphics[width=\linewidth]{../figures/H2-expspectrum.pdf}
    \end{center}
    \caption{
    Experimental photoionization spectrum of H$_2$-H$_2^+$ from Ref.~\cite{berkowitz1973comparison}
    }
    \label{fig:h2_spectrum}
\end{figure}

\subsubsection*{Task \#2}

You are provided with a \texttt{C++} code \texttt{FC.cxx} created by P.-N. Roy \cite{yang1995structure}, which calculates the photoionization spectrum for any molecule up to triple excitations and goes beyond the harmonic oscillator approximation. The theory is based on the paper by Ref~\cite{doktorov1977dynamical}. The molecule you will be investigating is $V_3$.
This code takes as input a file which requires the results of diagonalizing the mass-weighted hessian/force-constant matrix (2nd derivative of the Hamiltonian with respect to position). This input file is provided for you (V3).

Browse the following references to understand how the code works, as you will need a basic understanding of this for the next task, but it's not necessary to fully understand it. Compile and run the code \texttt{./FC\_quick V3}. This code outputs the spectrum \texttt{V3.spec.out}. Plot it in your favourite plotting program. 

\subsubsection*{Task \#3}

You are provided with a \texttt{python} notebook code Sample\_Vibronic.py which calculates all of the information you require.
\noindent This code takes as input a file which requires the following information:
\begin{enumerate}
    \item Number of atoms in the molecule
    \item Vibrational frequencies of the molecule in the ground electronic state
    \item Vibrational frequencies of the molecule in the excited electronic state
    \item Duschinsky Matrix (encodes information on transformation between ground and excited electronic states)
    \item Displacement vector
\end{enumerate}

\noindent This code outputs the spectrum in HTML format.

However, to be able to use this code, you require an input file. To create this input file, you will need to generate results from another code, \texttt{FC.cxx}. This code calculates the Franck-Condon Factors using matrix elements and recursive Hermite polynomial relations. Feel free to look around this code, but it is not necessary to completely understand it to achieve this task (the reference(s) noted in this code are Refs~\cite{yang1995structure,doktorov1977dynamical,quesadaFranckCondonFactorsCounting2019}. Each piece of information that you need to output has been clearly marked in the code and your task is to write that information to a file, which will then be used as your input file to \texttt{Sample\_Vibronic.py}. Once you have that input file, you should be able to produce the spectrum for $V_3$. Compare this spectrum to the previous method. What happens if you decrease the number of samples to 10? 100? 1000? At what number of samples do you feel the spectrum is converged?

\section*{Challenges}

\begin{enumerate}
    \item An alternative and analogous method to calculating these Franck-Condon Factors using matrix elements is to use a loop hafnian approach. This loop hafnian approach uses Gauss Boson Sampling which would allow these Factors to be calculated using a quantum circuit. Use the result of Task 3 to provide data to a skeleton code provided that uses loop hafnians to calculate the Franck-Condon Factors.
    \item Explain briefly the similarities and differences between these three methods.
\end{enumerate}

\section*{Possible Business Outcomes}

\begin{enumerate}
    \item Explain to a layperson what theoretical chemistry/physics is, in the general context of Franck-Condon Factors
    \item What is the importance of theoretical chemistry/physics from an economic point of view
    \item Explain to a layperson what a quantum circuit is and it's relationship to theoretical chemistry/physics
    \item What are advantages and disadvantages of codes licensed for the public domain and those that are licensed for private use
\end{enumerate}

\newpage

\bibliography{refs}
\bibliographystyle{unsrt}

\end{document}
